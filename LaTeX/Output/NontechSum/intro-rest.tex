
  A tempting interpretation of the discrepancy between the models' predictions and people's actual choices is that most people are just making a mistake in not investing more in stocks. Another possibility, though, is that the models are still missing some vitally important (and rational) factor that weighs on people's actual decisions; in this case, people might be making rational decisions and the models might be wrong.

  There's an obvious candidate for the missing factor.  For most households homeownership is the biggest financial decision in their lives.  Homeownership \textit{should} matter for consumers' choices about how willing they should rationally be to expose themselves to risky financial assets, for at least two reasons.  First, homeownership exposes consumers to housing market price risk, which should have the effect of reducing their appetite for being exposed to other kinds of risk.  Second, homeownership is associated with certain payment obligations (not just mortgages, but property taxes, maintenance costs, and so on), which reduces the flexibility they may have in adjusting their spending in response to income fluctuations.

  These points may seem obvious, but there is a good reason they have not been incorporated in previous analyses of households' optimal choice:  Taking account of these complexities greatly increases the computational difficulty of calculating  optimal decisions.  This technical report describes results obtained using the latest tools to be added to the \href{https://econ-ark.org/}{Econ-ARK} toolkit; with these tools, it should be much easier for economists, financial planners, and others to understand the appropriate role of homeownership in modifying investors' optimal saving and financial choices.

